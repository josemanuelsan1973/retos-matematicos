\documentclass[11pt]{article}
% decent example of doing mathematics and proofs in LaTeX.
% An Incredible degree of information can be found at
% http://en.wikibooks.org/wiki/LaTeX/Mathematics

% Use wide margins, but not quite so wide as fullpage.sty
\marginparwidth 0.5in 
\oddsidemargin 0.25in 
\evensidemargin 0.25in 
\marginparsep 0.25in
\topmargin -1in 
\textwidth 6in \textheight 9.6in
% That's about enough definitions

\usepackage{amsmath}
\usepackage{upgreek}
\usepackage[utf8]{inputenc}
\usepackage{amsfonts}
\usepackage{amsmath}
\usepackage{mathpazo}
\usepackage{multicol}
\usepackage{amssymb}
\usepackage{amsthm}
\usepackage{setspace}
\usepackage{fancybox}
\usepackage[spanish]{babel}
\usepackage[T1]{fontenc}
%\usepackage[pdftex]{graphicx}
\usepackage{cancel}
\usepackage{fancyhdr}
\usepackage{eurosym}
\usepackage[dvips]{graphicx}
%\usepackage{arcs}
%\DeclareGraphicsExtensions{.png,.pdf,.jpg}
\usepackage[colorlinks=true,linkcolor=blue]{hyperref}
\usepackage{pstricks-add}
\usepackage{auto-pst-pdf}
\usepackage[colorlinks=true]{hyperref}
\hypersetup{
    colorlinks,%
    citecolor=black,%
    filecolor=black,%
    linkcolor=black,%
    urlcolor=black,
%   hyperindex,
    breaklinks
}
\usepackage{polynom}
\usepackage{verbatim}
\parskip=1.5mm
\usepackage{wrapfig}

\definecolor{verdeodi}{RGB}{53,113,105}
\usepackage[most]{tcolorbox}
\newtcolorbox{fancybox}[1][]{
	enhanced,
	boxrule=1pt,arc=8pt,boxsep=0pt,
	left=0.8em,right=0.8em,top=1ex,bottom=1ex,colback=verdeodi!20,colframe=verdeodi,#1
}






\makeatletter%%%%%%%%para referenciar ecuaciones en línea
\newcommand*{\inlineequation}[2][]{%
  \begingroup
    % Put \refstepcounter at the beginning, because
    % package `hyperref' sets the anchor here.
    \refstepcounter{equation}%
    \ifx\\#1\\%
    \else
      \label{#1}%
    \fi
    % prevent line breaks inside equation
    \relpenalty=10000 %
    \binoppenalty=10000 %
    \ensuremath{%
      % \displaystyle % larger fractions, ...
      #2%
    }%
    \hfill~\@eqnnum %%%%si quieres que aparezca al lado de la ecuación quita \hfill
  \endgroup
}
\makeatother

%%%%% Genera los pies de Figuras y Tablas en itálicas y a tamaño footnote
\usepackage[format=plain, labelformat=simple, labelsep=period, center]{caption}
\renewcommand\thefigure{\arabic{figure}} % Genera numeración X
\renewcommand\thetable{\arabic{table}} % Genera numeración X

\title{Retos Matemáticos
}
\author{\href{mailto:jmanuel.sanchez@educarex.es}{José Manuel Sánchez Muñoz}}
\date{\today}

\begin{document}
\renewcommand{\tablename}{Tabla} 

\maketitle

\thispagestyle{empty}

\begin{fancybox}%%%%%%%%%%%%%%%%%%%%%%%%%%%%%%%%%%%%
\noindent\textbf{Ejercicio:} Obténgase el valor de $x$.
\[\left(x-\frac{1}{x}\right)^{^1\!\!/\!_2}+\left(1-\frac{1}{x}\right)^{^1\!\!/\!_2}=x\]
\end{fancybox}%%%%%%%%%%%%%%%%%%%%%%%%%%%%%%%%%%%%%

En primer lugar vamos a operar en la ecuación intentando eliminar las raíces cuadradas iniciales, para ello se eleva al cuadrado ambos miembros,
\[\left(x-\frac{1}{x}\right)+\left(1-\frac{1}{x}\right)+2\sqrt{\left(x-\frac{1}{x}\right)\left(1-\frac{1}{x}\right)}=x^2\]
\[x-\frac{1}{x}+1-\frac{1}{x}+2\sqrt{\frac{(x^2-1)\cdot(x-1)}{x^2}}=x^2\]
\[x-\frac{1}{x}+\frac{2\sqrt{x^3-x^2-x+1}}{|x|}+1-\frac{1}{x}=x^2\]

Podemos considerar resolver la ecuación con $x<0$ o con $x\geq 0$. Se considera esta segunda opción,
\[x-\frac{2}{x}+\frac{2\sqrt{x^3-x^2-x+1}}{x}+1=x^2\]

Consideramos ahora la $x$ como denominador común,
\[x^2-2+2\sqrt{x^3-x^2-x-1}+x=x^3\]

Se deja la raíz en el miembro de la izquierda,
\[2\sqrt{x^3-x^2-x+1}=x^3-x^2-x+2\]

En este punto, tuve la idea ``feliz'' de realizar el cambio de varible $x^3-x^2-x+1=a^2$, por lo tanto $x^3-x^2-x+2=a^2+1$, resultando entonces,
\[2a=a^2+1\Rightarrow a^2-2a+1=0\Rightarrow (a-1)^2=0\Rightarrow a=1\]

Se procede entonces a deshacer el cambio de variable, resultando
\[x^3-x^2-x+1=1\Rightarrow x(x^2-x-1)=0\Rightarrow \begin{cases}x=0\\[2mm]x^2-x-1=0\end{cases}\]
\[x^2-x-1=0\Rightarrow x=\frac{1\pm\sqrt{1+4}}{2}\Rightarrow \begin{cases}x=\dfrac{1+\sqrt{5}}{2}=\phi\approx 1.618\\[2mm]x=\dfrac{1-\sqrt{5}}{2}<0\Rightarrow \textrm{No vale pues }x\geq0\end{cases}\]

Por lo tanto la solución buscada resulta $\boxed{x=\phi\approx 1.618}$ denominado \emph{número áureo}.





\end{document}
